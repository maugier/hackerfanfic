\documentclass[a4paper,12pt,oneside]{memoir}

\usepackage[french]{babel}
\usepackage{hackerfanfic}
\usepackage{lipsum}
%\usepackage{lua-visual-debug}

\title{Alice et Bob a Cryptoland}
\newcommand{\subtitle}{Tome 1 -- Asymétrique}
\author{M. Roboto}
\date{2020}
\newcommand{\thepublisher}{Editions 1337}


\begin{document}

\maketitle
\tableofcontents*

\chapter{Un nouvel espoir pour RSA}

\begin{quote}[Arjen Lenstra]
	\enquote{Ron was Wrong, Whit is Right.}
\end{quote}	

\lettrine{L}{orem} \lipsum[1-4]

\begin{chatlog}
	\say[Alice] Salut, chéri, c'est une clef publique dans ton pantalon ou tu es juste content de me voir ?
	\think[Bob] Hein? Qu'est-ce qu'elle me veut, celle-là, encore?
	\say[Bob] Désolé, je connais pas ce protocole.
	\say[Alice] Pfft. Tu ne disais pas ça hier soir pendant l'échange de clefs.
	\think[Alice] Et voila, encore un qui perd ses clefs de session partout.
\end{chatlog}

\lipsum[5-7]


\chapter{La Courbe elliptique contre-attaque}

\lettrine{A}{insi} commença le deuxième chapitre, qui allait être encore plus
époustouflant que le premier, parce que voila, même que genre alors bon.

\lipsum

\chapter{Le Retour à la libc}

\lettrine{E}{t} ce chapitre est vide parce que c'est déja bien assez long.

\end{document}
